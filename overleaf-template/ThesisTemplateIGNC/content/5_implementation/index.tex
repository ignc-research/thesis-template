\chapter{Implementation}
This section will show in more detail how the conceptual design was implemented. First, the already existing simulation modules of arena-rosnav upon which this project’s work depends will be introduced. Then the data generation, behavior cloning, scenario generation and deep reinforcement learning modules will be presented.

The simulation environment in arena-rosnav is flatland. It manages the map as well as the positions and velocities of the robot and obstacles. It is integrated with ROS, so information about the state of the simulation is published and consumed over ROS topics. In particular, we will be interested in the following topics:

\begin{itemize}
    \item /cmd\_vel: publishes the velocity of the robot.
    \item /odom: publishes current position and pose of the robot.
    \item /scan: publishes current laser scan.
    \item /move\_base\_simple/goal: publishes the current coordinates of the subgoal. Consumed by move\_base planner like MPC. This was used for recording MPC data.
    \item /subgoal: publishes the current coordinates of the subgoal. Consumed by DRL planners. This was used for recording human expert data.
\end{itemize}

Another important component built into arena-rosnav is the ObservationCollector, which makes observation data available to other modules. It subscribes to the ROS topics above and can return synchronized laser scan and goal in robot pose observations via the function get\_observations(). It interfaces with the recording module via FlatlandEnv, an OpenAI Gym environment.

Finally, the Task module generates the navigation task for each recording and training episode, which can be random tasks or scenario tasks. Random tasks consist of a random starting position of the robot, a random goal as well as random positions for the dynamic obstacles. The number of dynamic obstacles is determined by the current stage in the training curriculum. Scenario tasks are generated according to a specification stored in a json file. This includes specific starting and goal positions of the robot and positions of the dynamic obstacles. Additionally, it is possible to specify waypoints that determine exactly where the dynamic obstacles move and a wait timer, which makes the dynamic obstacle stand in place before beginning its motion along the waypoints. The wait timer will be particularly important in the experiments with training specific behaviors.

\section{Data Generation}
In the data generation section of the conceptual design, there are three modules: expert, simulation and recording. As discussed above, the simulation environment is already implemented in arena-rosnav. The implementation of the expert and recording modules depends on the type of expert being recorded. The central problems to be solved in each case are implementing a control loop and synchronizing the observation data sources. 

\section{Behavior Cloning}
The behavior cloning module is implemented in two files: dataset.py and pretrain.py. Both are implemented in PyTorch, a Python deep learning library which was also used to implement stable-baselines3, the library of DRL training algorithms used in arena-rosnav.


\section{Proximal Policy Optimization}
The DRL algorithm used in arena-rosnav is Proximal Policy Optimization (PPO). For this paper, the key aspect of PPO is that it is an on-policy algorithm \citep{ppo}. On-policy means that the network being trained is the same one that is used to gather training data in the form of episodes in the environment \citep{lapan}. As a result, after each training iteration, the episodes must be discarded and fresh training data must be collected. This is because the old training data was generated by a policy that is, generally, worse than the one currently being trained \citep{lapan}.
This leads to a high sample inefficiency. A possible solution could be first training on expert demonstrations, which are episodes that show the targeted behavior of the agent. This area of research is Imitation Learning (IL) and is the topic of the following section.

\section{Imitation Learning}
Imitation Learning is an area of research that attempts to train agents to imitate an expert’s behavior based on a set of demonstrations \citep{abbeel}. It requires a method of recording expert demonstrations and a method for training agents on them \citep{abbeel}. Recording the demonstrations can take many forms, such as videos showing the expert or values of joysticks in teleoperation scenarios \citep{abbeel} or the observations and actions of a robot in arena-rosnav.

