\chapter{Implementation}
This section will show in more detail how the conceptual design was implemented. First, the already existing simulation modules of arena-rosnav upon which this project’s work depends will be introduced. Then the data generation, behavior cloning, scenario generation and deep reinforcement learning modules will be presented.

The simulation environment in arena-rosnav is flatland. It manages the map as well as the positions and velocities of the robot and obstacles. It is integrated with ROS, so information about the state of the simulation is published and consumed over ROS topics. In particular, we will be interested in the following topics:

\begin{itemize}
    \item /cmd\_vel: publishes the velocity of the robot.
    \item /odom: publishes current position and pose of the robot.
    \item /scan: publishes current laser scan.
    \item /move\_base\_simple/goal: publishes the current coordinates of the subgoal. Consumed by move\_base planner like MPC. This was used for recording MPC data.
    \item /subgoal: publishes the current coordinates of the subgoal. Consumed by DRL planners. This was used for recording human expert data.
\end{itemize}

Another important component built into arena-rosnav is the ObservationCollector, which makes observation data available to other modules. It subscribes to the ROS topics above and can return synchronized laser scan and goal in robot pose observations via the function get\_observations(). It interfaces with the recording module via FlatlandEnv, an OpenAI Gym environment.

Finally, the Task module generates the navigation task for each recording and training episode, which can be random tasks or scenario tasks. Random tasks consist of a random starting position of the robot, a random goal as well as random positions for the dynamic obstacles. The number of dynamic obstacles is determined by the current stage in the training curriculum. Scenario tasks are generated according to a specification stored in a json file. This includes specific starting and goal positions of the robot and positions of the dynamic obstacles. Additionally, it is possible to specify waypoints that determine exactly where the dynamic obstacles move and a wait timer, which makes the dynamic obstacle stand in place before beginning its motion along the waypoints. The wait timer will be particularly important in the experiments with training specific behaviors.

% Checked with grammarly
\section{Supervised Learning}
Supervised learning is the most common technique used to train neural networks, where a non-linear function between inputs and outputs is learned from examples. Following \citep{lapan}, we will outline the basics of supervised learning.
\\\\\noindent 
Using a training dataset of labeled examples, pairs of inputs, x, and outputs, y, neural network (NN) parameters are optimized so that the output of the NN for each input is as close as possible to the correct output. In this project, the outputs which will be predicted are the actions of the robot, and the inputs are the observations of the robot. These are continuous quantities, which makes this a regression problem.
\\\\\noindent 
How close the output is to the label is measured using the loss function. In this project, the mean-squared error loss will be used, which is the most common choice for regression problems:
\[MSE = \dfrac{1}{N}\sum_{i=1}^{i=N}(y_{i, predicted} – y_{i, labelled})^2\]
Typically, this loss is computed in mini-batches, small numbers of samples drawn from the training dataset. The parameters are then optimized using stochastic gradient descent to minimize the loss function.
\\\\\noindent 
A potential problem with this procedure is overfitting, where the network learns to predict the training set well but fails to generalize to unseen examples. The solution is to use a test set, which is not used to optimize the networks, but is used to monitor the performance on unseen examples.
\\\\\noindent 
The test and training losses are used in early stopping to avoid overfitting. This is a common technique, where training is stopped if the test loss starts increasing while the training loss continues to drop. This indicates that the network is overfitting to details of the training set which are reducing effectiveness in the test set.

\section{Behavior Cloning}
The behavior cloning module is implemented in two files: dataset.py and pretrain.py. Both are implemented in PyTorch, a Python deep learning library which was also used to implement stable-baselines3, the library of DRL training algorithms used in arena-rosnav.


\section{Generating Specific Scenarios}
A specific scenario was generated using the arena-tools \citep{tools} Scenario Editor GUI, with the map drawn in Microsoft Paint.

\section{Continued Deep Reinforcement Learning}
For continued DRL with pretrained weights, arena-rosnav’s DRL training script, train\_agent.py, was used.


