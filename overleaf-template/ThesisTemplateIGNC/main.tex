\newcommand{\mypapersize}{A4}
%% e.g., "A4", "letter", "legal", "executive", ...
%% The size of the paper of the resulting PDF file.

\newcommand{\mylaterality}{oneside}
%% "oneside" or "twoside"
%% Either you are creating a document which is printed on both, left pages
%% and right pages (twoside) or you create a document which is printed
%% on right pages only (oneside).

\newcommand{\mydraft}{false}
%% "true" or "false"
%% Use draft mode? If true, included graphics are replaced by empty
%% rectangles (of same size) and overfull boxes (in margin space) are
%% marked with black box (-> easy to spot!)

\newcommand{\myparskip}{no}
\setlength{\parindent}{2em}
%% e.g., "no", "full", "half", ...
%% How to separate paragraphs: indention ("no") or spacing ("half",
%% "full", ...).

\newcommand{\myBCOR}{0mm}
%% Inner binding correction. This value depends on the method which is
%% being used to bind your printed result. Some techniques do not
%% require a binding correction at all ("0mm"), other require for
%% example "5mm". Refer to KOMA script documentation for a detailed
%% explanation what a binding correction is and how to measure it.

\newcommand{\myfontsize}{12pt}
%% e.g., 10pt, 11pt, 12pt
%% The font size of the main text in pt (points).

\newcommand{\mylinespread}{1.0}
%% e.g., 1.0, 1.5, 2.0
%% Line spacing in %/100. For example 1.5 means 150% of the usual line
%% spacing. Please use with caution: 100% ("1.0") is fine because the
%% font was designed for it.

\newcommand{\mylanguage}{ngerman,american}
%% "english,ngerman", "ngerman,english", ...
%% NOTE: The *last* language is the active one!
%% See babel documentation for further details.

%% BibLaTeX-settings: (see biblatex reference for further description)
\newcommand{\mybiblatexstyle}{numeric}
%% e.g., "alphabetic", "authoryear", ...
%% The biblatex style which is being used for referencing. See
%% biblatex documentation for further details and more values.
%%
%% CAUTION: if you change the style, please check for (in)compatible
%%          "biblatex" package options in the file
%%          "template/preamble.tex"! For example: "alphabetic" does
%%          not have an option "dashed=..." and causes an error if it
%%          does not get removed from the list of options.

\newcommand{\mybiblatexdashed}{false}  %% "true" or "false"
%% If true: replace recurring reference authors with a dash.

\newcommand{\mybiblatexbackref}{true}  %% "true" or "false"
%% If true: create backward links from reference to citations.

\newcommand{\mybiblatexfile}{references-biblatex.bib}
%% Name of the biblatex file that holds the references.

\newcommand{\mydispositioncolor}{0,0,0}
%% e.g., "30,103,182" (blue/turquois), "0,0,0" (black), ...
%% Color of the headings and so forth in RGB (red,green,blue) values.
%% NOTE: if you are using "0,0,0" for black, printers might still
%%       recognize pages as color pages. In case this is a problem
%%       (paying for color print-outs vs. paying for b/w-printouts)
%%       please edit file "template/preamble.tex" and change
%%       "\definecolor{DispositionColor}{RGB}{\mydispositioncolor}"
%%       to "\definecolor{DispositionColor}{gray}{0}" and thus
%%       overwriting the value of \mydispositioncolor above.

\newcommand{\mycolorlinks}{true}  %% "true" or "false"
%% Enables or disables colored links (hyperref package).

\newcommand{\mytitlepage}{template/title_Thesis_TU_Graz}
%\newcommand{\mygermantitlepage}{template/title_Thesis_TU_Graz_German}
%% Your own or one of following pre-defined title pages:
%% "template/title_plain_maketitle": simple maketitle page
%% "template/title_Diplomarbeit_KF_Uni_Graz.tex": fancy (german) title page for KF Uni Graz
%% "template/title_Thesis_TU_Graz":
%%             titlepage for Graz University of Technology (correct
%%             (old?) Corporate Design) by Karl Voit (2012)
%% "template/title_Thesis_TU_Graz_-_kazemakase":
%%             titlepage for Graz University of Technology
%%             (correct new Corporate Design) by kazemakase (2013):
%%             see https://github.com/novoid/LaTeX-KOMA-template/issues/5
%% "template/title_VWA": titlepage for Vorwissenschaftliche Arbeit


%% Load main settings for document preamble:
\input{template/preamble}%% DO NOT REMOVE THIS LINE!


\setboolean{english_affidavit}{true}  %% "true" or "false"
%% If set to "true": the language of the statutory declaration text is set to
%% English, otherwise it is in German.


%% ========================================================================
%%%% Document metadata
%% ========================================================================

%% general metadata:
\newcommand{\myauthor}{Author Name}  %% also used for PDF metadata (hyperref)
\newcommand{\myauthorwithexistingtitles}{\myauthor{}, MSc}  %% including
                                %% university degree already held
                                %% (BSc, MSc, ...)
\newcommand{\mytitle}{Development of Behavior Cloning Approaches for DRL}  %% also used for PDF metadata (hyperref)
\newcommand{\mytitlegerman}{German Thesis Title}
\newcommand{\mysubject}{SUBJECT}  %% also used for PDF metadata (hyperref)
\newcommand{\mykeywords}{KEYWORDS}  %% also used for PDF metadata (hyperref)

%% this information is used only for generating the title page:
\newcommand{\myworktitle}{Bachelor Thesis}  %% official type of work like ``Master theses''
\newcommand{\myworktitlegerman}{Masterarbeit}
\newcommand{\mygrade}{Bachelor of Science} %% title you are getting with this work like ``Master of ...''
\newcommand{\mygradegerman}{Diplom-Ingenieur}
\newcommand{\mystudy}{Computer Science} %% your study like ``Arts''
\newcommand{\mydegreeprogramme}{\mystudy} %% Master's or PhD degree programme
\newcommand{\mydegreeprogrammegerman}{Masterstudium: \mystudy} %% Master's or PhD degree programme
\newcommand{\myuniversity}{Technische Universität Berlin} %% your university/school
\newcommand{\myuniversitygerman}{Technische Universität Graz} %% your university/school
\newcommand{\myinstitute}{Industry Grade Networks and Clouds} %% affiliation
\newcommand{\myinstitutehead}{Head of Institute} %% head of institute
\newcommand{\mysupervisor}{M.Sc. Linh Kästner} %% your supervisor
\newcommand{\myfirstexaminer}{Prof. Dr.-Ing. Jens Lambrecht}
\newcommand{\mysecondexaminer}{Prof. Dr.-Ing. Sebastian Möller}
\newcommand{\mycosupervisors}{Co-supervisor 1, Co-supervisor 2}
\newcommand{\myevaluator}{Prof.~Some Genius} %% your evaluator
\newcommand{\myhomestreet}{Street} %% your home street (with house number)
\newcommand{\myhometown}{Home town} %% your home town
\newcommand{\myhomepostalnumber}{post} %% your postal number of home town
\newcommand{\mysubmissionmonth}{month} %% month you are handing in
\newcommand{\mysubmissionyear}{year} %% year you are handing in
\newcommand{\mysubmissiontown}{\myhometown} %% town of handing in (or \myhometown)

%% additional information for generic_documentation title page
\newcommand{\myid}{1234567} %% Matrikelnummer
\newcommand{\mylecture}{LECTURE} %%


%% ========================================================================
%%%% MISC command definitions
%% ========================================================================
\input{template/mycommands}

%% ========================================================================
%%%% Typographic settings
%% ========================================================================
\input{template/typographic_settings}


%% ========================================================================
%%%% MISC usepackages
%% ========================================================================

%% ... it's OK to put here your own usepackage commands ...
%test

\usepackage{soul}
\usepackage{booktabs}
\usepackage{amsmath}
\usepackage{placeins}
\usepackage{geometry}
\geometry{a4paper, portrait, tmargin=1.5in, bmargin=1.5in, outer=1.5in}

%% ========================================================================
%%%% MISC self-defined commands and settings
%% ========================================================================

%% ... it's OK to put here your own newcommand/newenvironment-definitions ...




\newcommand{\myLaT}{\LaTeX{}@TUG\xspace} %% LaTeX@TUG text "logo"

\hyphenation{ex-am-ple hy-phen-ate}  %% in order to use German umlauts
%% here (Ver-\"of-fent-li-chung), you have to check for
%% activated \usepackage[T1]{fontenc} in the preamble

%% override default language of babel: (be sure to know, what you're
%% doing here)
%\selectlanguage{american}
%\selectlanguage{ngerman}

%% ========================================================================
%%%% Templates
%% ========================================================================

%% template for inserting figures:
% \myfig{}%% filename
%       {}%% width/height
%       {}%% caption
%       {}%% optional (short) caption for list of figures
%       {fig:}%% label

%% acronyms in small caps: \myacro{UNESCO}

\input{template/pdf_settings}  %% should be *last* definitions in preamble!
%% ========================================================================
%%%% begin{document}
%% ========================================================================
\usepackage{float}
\begin{document}

\frontmatter                    %% KOMA: roman page numbers and such; only available in scrbook

\input{\mytitlepage}
%\input{\mygermantitlepage}

%%%% Time-stamp: <2017-02-14 16:01:12 vk>
%% ========================================================================
%%%% Disclaimer
%% ========================================================================
%%
%% created by
%%
%%      Karl Voit, and Matthias Wölbitsch
%%
%%
%% code for the date and signature fields adapted from
%% http://tex.stackexchange.com/a/20562


\newcommand{\textfield}[2]{
  \vbox{
    \hsize=#1\kern3cm\hrule\kern1ex
    \hbox to \hsize{\strut\hfil\footnotesize#2\hfil}%chktex 1
  }
}

\chapter*{Affidavit}
\foreignlanguage{ngerman}{%
I hereby declare that the thesis submitted is my own unaided work. All direct or indirect sources used are acknowledged as references.

\noindent\emph{Hiermit erkläre ich, dass ich die vorliegende Arbeit selbstständig und eigenhändig sowie ohne unerlaubte fremde Hilfe und ausschließlich unter Verwendung der aufgeführten Quellen und Hilfsmittel angefertigt habe.}}


\textfield{4cm}{Berlin, August 19, 2021}%chktex 1

\newpage
\newpage\null\thispagestyle{empty}\newpage




%% include the abstract without chapter number but include it on table of contents:
\cleardoublepage
\chapter*{Abstract}\label{cha:abstract}
Deep Reinforcement Learning has emerged as a promising framework for autonomous robotic navigation. It has been used to achieve state-of-the-art performance particularly in the realm of collaborative environments, where many dynamic obstacles must be avoided. Its limitations include high sample inefficiency and long training sessions, even when performed on powerful hardware. This thesis investigates the potential for using behavior cloning to mitigate these limitations. In particular, a behavior cloning pipeline for recording expert demonstrations will be implemented and used to perform a series of experiments with the aim of reducing training time.
Inspired by related works that have used behavior cloning to teach robots specific skills, we will also perform an experiment to teach a robot to wait for an obstacle to clear a blocked corridor.

%\newpage\null\thispagestyle{empty}\newpage

\KOMAoptions{open=left}
\chapter*{Kurzfassung}\label{cha:abstract_german}
Deep Reinforcement Learning hat sich zu einer viel versprechenden Methode für autonome Roboternavigation entwickelt und wird in Umgebungen erforscht, in denen dynamisch bewegliche Hindernisse mit höchster Präzision erkannt und umgangen werden müssen.
Diese Methode stößt jedoch aufgrund sehr ineffizienter Datennutzung und langen Trainingseinheiten – selbst bei sehr leistungsfähiger Hardware – an ihre Grenzen.
In dieser Arbeit wird nun untersucht, ob die Verwendung von Behavior Cloning diese Nachteile abmildern kann.
Insbesondere wird dazu eine Behavior Cloning Pipeline zur Aufzeichnung der Expert Demonstrations  implementiert.
Damit wird eine Reihe von Experimenten durchgeführt mit dem Ziel, die Trainingszeiten zu reduzieren.
Inspiriert durch verwandte Arbeiten, in denen Robotern durch Verwendung von Behavior Cloning spezielle Fähigkeiten beigebracht wurden, werden wir auch ein Experiment durchführen, in dem der Roboter lernen soll, vor einem Hindernis zu warten, bis dieses einen blockierten Korridor verlassen hat.

\KOMAoptions{open=right}

%\newpage\null\thispagestyle{empty}\newpage




\tableofcontents

\chapter*{List of Abbreviations}
\addcontentsline{toc}{chapter}{List of Abbreviations}
\begin{table}[htb]
\centering
    \begin{tabular}{ @{}p{2cm}l@{} } 
    \textbf{BC} & Behavior Cloning\\
    \textbf{DRL} & Deep Reinforcement Learning\\
    \textbf{IL} & Imitation Learning\\
    \textbf{IRL} & Inverse Reinforcement Learning\\
    \textbf{NN} & Neural Network\\
    \textbf{PPO} & Proximal Policy Optimization\\
    \textbf{RL} & Reinforcement Learning\\
    \textbf{ROS} & Robot Operating System\\
    \end{tabular}
\end{table}
\listoffigures
\listoftables
%\newpage\null\thispagestyle{empty}\newpage % New Page


\mainmatter                     %% KOMA: marks main part using arabic page numbers and such; only available in scrbook

\chapter{Conceptual Design (BA: at least 2-3pages, MA at least 3 pages)}
This section introduces the conceptual design of the project. Starting from the goal, the necessary modules and their interactions will be presented in a system design flowchart, and additional requirements for the implementation will be considered. Finally, an overview of the observation and action spaces of the robot are given to illustrate the data that will be recorded in demonstrations.

% Checked with grammarly
\section{Supervised Learning}
Supervised learning is the most common technique used to train neural networks, where a non-linear function between inputs and outputs is learned from examples. Following \citep{lapan}, we will outline the basics of supervised learning.
\\\\\noindent 
Using a training dataset of labeled examples, pairs of inputs, x, and outputs, y, neural network (NN) parameters are optimized so that the output of the NN for each input is as close as possible to the correct output. In this project, the outputs which will be predicted are the actions of the robot, and the inputs are the observations of the robot. These are continuous quantities, which makes this a regression problem.
\\\\\noindent 
How close the output is to the label is measured using the loss function. In this project, the mean-squared error loss will be used, which is the most common choice for regression problems:
\[MSE = \dfrac{1}{N}\sum_{i=1}^{i=N}(y_{i, predicted} – y_{i, labelled})^2\]
Typically, this loss is computed in mini-batches, small numbers of samples drawn from the training dataset. The parameters are then optimized using stochastic gradient descent to minimize the loss function.
\\\\\noindent 
A potential problem with this procedure is overfitting, where the network learns to predict the training set well but fails to generalize to unseen examples. The solution is to use a test set, which is not used to optimize the networks, but is used to monitor the performance on unseen examples.
\\\\\noindent 
The test and training losses are used in early stopping to avoid overfitting. This is a common technique, where training is stopped if the test loss starts increasing while the training loss continues to drop. This indicates that the network is overfitting to details of the training set which are reducing effectiveness in the test set.

\section{Behavior Cloning}
The behavior cloning module is implemented in two files: dataset.py and pretrain.py. Both are implemented in PyTorch, a Python deep learning library which was also used to implement stable-baselines3, the library of DRL training algorithms used in arena-rosnav.


\section{Generating Specific Scenarios}
A specific scenario was generated using the arena-tools \citep{tools} Scenario Editor GUI, with the map drawn in Microsoft Paint.

\chapter{Conceptual Design (BA: at least 2-3pages, MA at least 3 pages)}
This section introduces the conceptual design of the project. Starting from the goal, the necessary modules and their interactions will be presented in a system design flowchart, and additional requirements for the implementation will be considered. Finally, an overview of the observation and action spaces of the robot are given to illustrate the data that will be recorded in demonstrations.

% Checked with grammarly
\section{Supervised Learning}
Supervised learning is the most common technique used to train neural networks, where a non-linear function between inputs and outputs is learned from examples. Following \citep{lapan}, we will outline the basics of supervised learning.
\\\\\noindent 
Using a training dataset of labeled examples, pairs of inputs, x, and outputs, y, neural network (NN) parameters are optimized so that the output of the NN for each input is as close as possible to the correct output. In this project, the outputs which will be predicted are the actions of the robot, and the inputs are the observations of the robot. These are continuous quantities, which makes this a regression problem.
\\\\\noindent 
How close the output is to the label is measured using the loss function. In this project, the mean-squared error loss will be used, which is the most common choice for regression problems:
\[MSE = \dfrac{1}{N}\sum_{i=1}^{i=N}(y_{i, predicted} – y_{i, labelled})^2\]
Typically, this loss is computed in mini-batches, small numbers of samples drawn from the training dataset. The parameters are then optimized using stochastic gradient descent to minimize the loss function.
\\\\\noindent 
A potential problem with this procedure is overfitting, where the network learns to predict the training set well but fails to generalize to unseen examples. The solution is to use a test set, which is not used to optimize the networks, but is used to monitor the performance on unseen examples.
\\\\\noindent 
The test and training losses are used in early stopping to avoid overfitting. This is a common technique, where training is stopped if the test loss starts increasing while the training loss continues to drop. This indicates that the network is overfitting to details of the training set which are reducing effectiveness in the test set.

\section{Behavior Cloning}
The behavior cloning module is implemented in two files: dataset.py and pretrain.py. Both are implemented in PyTorch, a Python deep learning library which was also used to implement stable-baselines3, the library of DRL training algorithms used in arena-rosnav.


\section{Generating Specific Scenarios}
A specific scenario was generated using the arena-tools \citep{tools} Scenario Editor GUI, with the map drawn in Microsoft Paint.

\chapter{Conceptual Design (BA: at least 2-3pages, MA at least 3 pages)}
This section introduces the conceptual design of the project. Starting from the goal, the necessary modules and their interactions will be presented in a system design flowchart, and additional requirements for the implementation will be considered. Finally, an overview of the observation and action spaces of the robot are given to illustrate the data that will be recorded in demonstrations.

% Checked with grammarly
\section{Supervised Learning}
Supervised learning is the most common technique used to train neural networks, where a non-linear function between inputs and outputs is learned from examples. Following \citep{lapan}, we will outline the basics of supervised learning.
\\\\\noindent 
Using a training dataset of labeled examples, pairs of inputs, x, and outputs, y, neural network (NN) parameters are optimized so that the output of the NN for each input is as close as possible to the correct output. In this project, the outputs which will be predicted are the actions of the robot, and the inputs are the observations of the robot. These are continuous quantities, which makes this a regression problem.
\\\\\noindent 
How close the output is to the label is measured using the loss function. In this project, the mean-squared error loss will be used, which is the most common choice for regression problems:
\[MSE = \dfrac{1}{N}\sum_{i=1}^{i=N}(y_{i, predicted} – y_{i, labelled})^2\]
Typically, this loss is computed in mini-batches, small numbers of samples drawn from the training dataset. The parameters are then optimized using stochastic gradient descent to minimize the loss function.
\\\\\noindent 
A potential problem with this procedure is overfitting, where the network learns to predict the training set well but fails to generalize to unseen examples. The solution is to use a test set, which is not used to optimize the networks, but is used to monitor the performance on unseen examples.
\\\\\noindent 
The test and training losses are used in early stopping to avoid overfitting. This is a common technique, where training is stopped if the test loss starts increasing while the training loss continues to drop. This indicates that the network is overfitting to details of the training set which are reducing effectiveness in the test set.

\section{Behavior Cloning}
The behavior cloning module is implemented in two files: dataset.py and pretrain.py. Both are implemented in PyTorch, a Python deep learning library which was also used to implement stable-baselines3, the library of DRL training algorithms used in arena-rosnav.


\section{Generating Specific Scenarios}
A specific scenario was generated using the arena-tools \citep{tools} Scenario Editor GUI, with the map drawn in Microsoft Paint.

\chapter{Conceptual Design (BA: at least 2-3pages, MA at least 3 pages)}
This section introduces the conceptual design of the project. Starting from the goal, the necessary modules and their interactions will be presented in a system design flowchart, and additional requirements for the implementation will be considered. Finally, an overview of the observation and action spaces of the robot are given to illustrate the data that will be recorded in demonstrations.

% Checked with grammarly
\section{Supervised Learning}
Supervised learning is the most common technique used to train neural networks, where a non-linear function between inputs and outputs is learned from examples. Following \citep{lapan}, we will outline the basics of supervised learning.
\\\\\noindent 
Using a training dataset of labeled examples, pairs of inputs, x, and outputs, y, neural network (NN) parameters are optimized so that the output of the NN for each input is as close as possible to the correct output. In this project, the outputs which will be predicted are the actions of the robot, and the inputs are the observations of the robot. These are continuous quantities, which makes this a regression problem.
\\\\\noindent 
How close the output is to the label is measured using the loss function. In this project, the mean-squared error loss will be used, which is the most common choice for regression problems:
\[MSE = \dfrac{1}{N}\sum_{i=1}^{i=N}(y_{i, predicted} – y_{i, labelled})^2\]
Typically, this loss is computed in mini-batches, small numbers of samples drawn from the training dataset. The parameters are then optimized using stochastic gradient descent to minimize the loss function.
\\\\\noindent 
A potential problem with this procedure is overfitting, where the network learns to predict the training set well but fails to generalize to unseen examples. The solution is to use a test set, which is not used to optimize the networks, but is used to monitor the performance on unseen examples.
\\\\\noindent 
The test and training losses are used in early stopping to avoid overfitting. This is a common technique, where training is stopped if the test loss starts increasing while the training loss continues to drop. This indicates that the network is overfitting to details of the training set which are reducing effectiveness in the test set.

\section{Behavior Cloning}
The behavior cloning module is implemented in two files: dataset.py and pretrain.py. Both are implemented in PyTorch, a Python deep learning library which was also used to implement stable-baselines3, the library of DRL training algorithms used in arena-rosnav.


\section{Generating Specific Scenarios}
A specific scenario was generated using the arena-tools \citep{tools} Scenario Editor GUI, with the map drawn in Microsoft Paint.

\chapter{Conceptual Design (BA: at least 2-3pages, MA at least 3 pages)}
This section introduces the conceptual design of the project. Starting from the goal, the necessary modules and their interactions will be presented in a system design flowchart, and additional requirements for the implementation will be considered. Finally, an overview of the observation and action spaces of the robot are given to illustrate the data that will be recorded in demonstrations.

% Checked with grammarly
\section{Supervised Learning}
Supervised learning is the most common technique used to train neural networks, where a non-linear function between inputs and outputs is learned from examples. Following \citep{lapan}, we will outline the basics of supervised learning.
\\\\\noindent 
Using a training dataset of labeled examples, pairs of inputs, x, and outputs, y, neural network (NN) parameters are optimized so that the output of the NN for each input is as close as possible to the correct output. In this project, the outputs which will be predicted are the actions of the robot, and the inputs are the observations of the robot. These are continuous quantities, which makes this a regression problem.
\\\\\noindent 
How close the output is to the label is measured using the loss function. In this project, the mean-squared error loss will be used, which is the most common choice for regression problems:
\[MSE = \dfrac{1}{N}\sum_{i=1}^{i=N}(y_{i, predicted} – y_{i, labelled})^2\]
Typically, this loss is computed in mini-batches, small numbers of samples drawn from the training dataset. The parameters are then optimized using stochastic gradient descent to minimize the loss function.
\\\\\noindent 
A potential problem with this procedure is overfitting, where the network learns to predict the training set well but fails to generalize to unseen examples. The solution is to use a test set, which is not used to optimize the networks, but is used to monitor the performance on unseen examples.
\\\\\noindent 
The test and training losses are used in early stopping to avoid overfitting. This is a common technique, where training is stopped if the test loss starts increasing while the training loss continues to drop. This indicates that the network is overfitting to details of the training set which are reducing effectiveness in the test set.

\section{Behavior Cloning}
The behavior cloning module is implemented in two files: dataset.py and pretrain.py. Both are implemented in PyTorch, a Python deep learning library which was also used to implement stable-baselines3, the library of DRL training algorithms used in arena-rosnav.


\section{Generating Specific Scenarios}
A specific scenario was generated using the arena-tools \citep{tools} Scenario Editor GUI, with the map drawn in Microsoft Paint.

\chapter{Conceptual Design (BA: at least 2-3pages, MA at least 3 pages)}
This section introduces the conceptual design of the project. Starting from the goal, the necessary modules and their interactions will be presented in a system design flowchart, and additional requirements for the implementation will be considered. Finally, an overview of the observation and action spaces of the robot are given to illustrate the data that will be recorded in demonstrations.

% Checked with grammarly
\section{Supervised Learning}
Supervised learning is the most common technique used to train neural networks, where a non-linear function between inputs and outputs is learned from examples. Following \citep{lapan}, we will outline the basics of supervised learning.
\\\\\noindent 
Using a training dataset of labeled examples, pairs of inputs, x, and outputs, y, neural network (NN) parameters are optimized so that the output of the NN for each input is as close as possible to the correct output. In this project, the outputs which will be predicted are the actions of the robot, and the inputs are the observations of the robot. These are continuous quantities, which makes this a regression problem.
\\\\\noindent 
How close the output is to the label is measured using the loss function. In this project, the mean-squared error loss will be used, which is the most common choice for regression problems:
\[MSE = \dfrac{1}{N}\sum_{i=1}^{i=N}(y_{i, predicted} – y_{i, labelled})^2\]
Typically, this loss is computed in mini-batches, small numbers of samples drawn from the training dataset. The parameters are then optimized using stochastic gradient descent to minimize the loss function.
\\\\\noindent 
A potential problem with this procedure is overfitting, where the network learns to predict the training set well but fails to generalize to unseen examples. The solution is to use a test set, which is not used to optimize the networks, but is used to monitor the performance on unseen examples.
\\\\\noindent 
The test and training losses are used in early stopping to avoid overfitting. This is a common technique, where training is stopped if the test loss starts increasing while the training loss continues to drop. This indicates that the network is overfitting to details of the training set which are reducing effectiveness in the test set.

\section{Behavior Cloning}
The behavior cloning module is implemented in two files: dataset.py and pretrain.py. Both are implemented in PyTorch, a Python deep learning library which was also used to implement stable-baselines3, the library of DRL training algorithms used in arena-rosnav.


\section{Generating Specific Scenarios}
A specific scenario was generated using the arena-tools \citep{tools} Scenario Editor GUI, with the map drawn in Microsoft Paint.

\chapter{Conclusion}
Deep reinforcement learning is a promising approach for autonomous robotic navigation, a technology with numerous applications. While it can lead to state-of-the-art navigation performance, it suffers from long training times \citep{kastner}.

\chapter{Future Work}
This work can be extended in a number of ways, to improve experimental results.


\printbibliography              %% remove, if using BibTeX instead of biblatex
% \include{further_ressources}  %% this is a suggestion: you have to create this file on demand


%List of Abbreviations

%\appendix                       %% closes main document, appendix follows until end; only available in book-classes
%\addpart*{Appendix}             %% adding Appendix to tableofcontents
\chapter{Appendix}\label{appdx1}% chktex 8
\begin{table}[!h]
 \centering
 \captionof{table}{training\_curriculum\_map1small}\label{table:training-curriculum}
 \begin{tabular}{|l|l|}
 \hline
 \textbf{Training Stage} & \textbf{Number of dynamic obstacles} \\ \hline
 1 & 0 \\ \hline
 2 & 4 \\ \hline
 3 & 6 \\ \hline
 4 & 8 \\ \hline
 5 & 10 \\ \hline
 \end{tabular}
 \bigskip
\end{table}

\begin{table}[!h]
 \centering
 \captionof{table}{Hyperparameters for MPC-pretrained DRL Training}\label{table:hyperparameters-mpc}
 \begin{tabular}{|l|l|}
 \hline
 \textbf{Hyperparameter} & \textbf{Value} \\ \hline
 robot & myrobot \\ \hline
 batch\_size & 24000 \\ \hline
 gamma & 0.99 \\ \hline
 n\_steps & 1000 \\ \hline
 ent\_coef & 0.005 \\ \hline
 learning\_rate & 0.0003 \\ \hline
 vf\_coef & 0.22 \\ \hline
 max\_grad\_norm & 0.5 \\ \hline
 gae\_lambda & 0.95 \\ \hline
 m\_batch\_size & 15 \\ \hline
 n\_epochs & 3 \\ \hline
 clip\_range & 0.22 \\ \hline
 train\_max\_steps\_per\_episode & 500 \\ \hline
 eval\_max\_steps\_per\_episode & 650 \\ \hline
 goal\_radius & 0.25 \\ \hline
 reward\_fnc & rule\_00 \\ \hline
 discrete\_action\_space & false \\ \hline
 normalize & false \\ \hline
 task\_mode & staged \\ \hline
 \end{tabular}
 \bigskip
\end{table}

\begin{table}[!h]
 \centering
 \captionof{table}{Hyperparameters for baseline and human expert-pretrained DRL Training}\label{table:hyperparameters-human}
 \begin{tabular}{|l|l|}
 \hline
 \textbf{Hyperparameter} & \textbf{Value} \\ \hline
 robot & myrobot \\ \hline
 batch\_size & 24000 \\ \hline
 gamma & 0.99 \\ \hline
 n\_steps & 1000 \\ \hline
 ent\_coef & 0.005 \\ \hline
 learning\_rate & 0.0003 \\ \hline
 vf\_coef & 0.22 \\ \hline
 max\_grad\_norm & 0.5 \\ \hline
 gae\_lambda & 0.95 \\ \hline
 m\_batch\_size & 15 \\ \hline
 n\_epochs & 3 \\ \hline
 clip\_range & 0.22 \\ \hline
 train\_max\_steps\_per\_episode & 600 \\ \hline
 eval\_max\_steps\_per\_episode & 650 \\ \hline
 goal\_radius & 0.25 \\ \hline
 reward\_fnc & rule\_04 \\ \hline
 discrete\_action\_space & false \\ \hline
 normalize & false \\ \hline
 task\_mode & staged \\ \hline
 \end{tabular}
 \bigskip
\end{table}

%%%% end{document}
\end{document}
%% vim:foldmethod=expr
%% vim:fde=getline(v\:lnum)=~'^%%%%\ .\\+'?'>1'\:'='
%%% Local Variables:
%%% mode: latex
%%% mode: auto-fill
%%% mode: flyspell
%%% eval: (ispell-change-dictionary "en_US")
%%% TeX--: "main"
%%% End:
